\chapter*{Introductions}
\addcontentsline{toc}{chapter}{Introductions}

In contemporary smart cities, the integration of technology and data is reshaping the way we live. However, concerns regarding air quality are escalating, necessitating innovative and accurate forecasting techniques for air pollutant concentrations. The dynamics of urban environments, coupled with the complexity of pollutant sources, pose a formidable challenge for traditional forecasting methods. This master's thesis explores the use of deep learning techniques to predict air pollutant concentrations, with the goal of improving the accuracy and flexibility of forecasting models in the dynamic environment of smart cities. This research aims to develop robust and efficient tools for proactive air quality management by utilizing advanced neural networks. The goal is to provide decision-makers with timely and actionable insights to promote healthier and more sustainable urban living environments.

\section*{Background and Motivation}

The impact of air pollutants on public health is a critical concern in urban areas. Therefore, it is imperative to adopt a proactive and comprehensive approach to managing air quality. This is especially crucial in the context of smart cities, where technological advancements seamlessly integrate into urban infrastructure, underscoring the need to monitor and address air quality issues. Equipped with an array of sensors and data-driven technologies, smart cities offer an unprecedented opportunity to continually monitor air quality in real-time. However, the sheer volume, diversity, and intricacy of data generated in these smart environments present a substantial challenge for conventional forecasting methods. To bridge this gap, the study employs deep learning techniques known for their proficiency in handling vast and complex datasets. The objective is to improve the accuracy and flexibility of models that forecast air pollutant concentrations. The data collected in smart cities is rich and varied, and coupled with the multifaceted nature of pollutant sources, it emphasizes the necessity for advanced computational methods to extract valuable insights. Consequently, this research aims to combine health-conscious urban planning with advanced technology. Deep learning will be used to analyze air pollutants concentrations in the context of smart cities.

\section*{Methodologies}

The study employs a comprehensive methodology to analyze real-world data sets, with the aim of uncovering patterns and dynamics. Three distinct data sets are examined, each presenting unique challenges. The first phase introduces these data sets, outlining their unique attributes and contextualizing subsequent analyses within their complexities. This lays the groundwork for exploratory data analysis, a crucial step in revealing inherent relationships and dependencies within the data sets.

Exploratory data analysis is a comprehensive examination of a dataset to uncover correlations among variables and gain a macro-level understanding. This initial exploration forms the basis for subsequent in-depth analysis. In our study, we utilized Auto-Correlation Function (ACF) plots to identify temporal dependencies and patterns in the data, going beyond surface-level examination.

In addition, we conduct a comprehensive seasonal component analysis to reveal periodic trends and variations that may affect the overall behavior of the dataset. This detailed exploration serves as the basis for developing and implementing model architectures, drawing inspiration from state-of-the-art methodologies in the field.

The model architectures tested, proposed, or implemented from the state-of-the-art are designed to capture nuanced patterns discovered during both exploratory and in-depth analyses. These architectures are tailored to adapt to the intricacies of each dataset, highlighting the importance of a flexible and robust modeling framework. The results are then analyzed.

To assess the performance and generalizability of the proposed models, tests were conducted using both augmented and non-augmented data. Augmentation techniques aim to increase the diversity of the dataset, providing the models with a broader perspective to learn from. Furthermore, we scrutinized the impact of historical data volume on model performance, considering variations in the amount of historical data to unravel the dynamics of information assimilation over time.

\section*{Thesis structure}

The first chapter thoroughly examines the importance of air quality in the context of smart cities. The concept of smart cities is introduced, illustrated with real examples, and the role of open data in sustainable development is highlighted. The following discussion explores pollution monitoring in smart cities, providing clarity on key pollutants and introducing the Air Quality Index. A specific case study on Beijing is presented to demonstrate the usefulness of pollutant monitoring systems. Additionally, existing research on forecasting air pollutants is reviewed.

Chapter 2 focuses on a detailed examination of the datasets essential for our research. We evaluate data availability from three sources while maintaining a clear and logical structure. Using exploratory data analysis, we aim to understand the intricacies of the Citypulse, Seoul, and Madrid datasets. The chapter concludes with a detailed analysis of each dataset, examining the seasonality and autocorrelation functions of each series, laying the foundation for subsequent modeling.

Chapter 3 discusses model development, with an emphasis on the crucial data preparation phase. This section covers various activities, including feature engineering, data scaling, augmentation, and splitting, which prepare for experiments with different model architectures. The chapter also presents metrics for evaluation and hyperparameters used during the training phase, as well as explaining the key aspects of the model architectures. 

Chapter 4 presents the performance results on the test sets for Citypulse, Seoul, and Madrid. The best models are comparatively analyzed, including a comparison with a baseline ARIMA model, error trends over time, and error analysis. 

Finally, the results are discussed, and suggestions for further developments in the field are provided.

This exploration aims to provide valuable insights, methodologies, and findings that can enhance our understanding of air quality in smart cities. The ultimate goal is to facilitate informed decision-making for improved urban living.